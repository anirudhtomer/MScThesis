% !TEX root =  ../../thesis.tex

\chapter{Analysis of blood donor data set}
\label{ch : blood_donor}
 
 In this chapter we will present the analysis of the blood donor data set \citep{nasserinejad_prevalence_2015} using the Bayesian heterogeneity model. The data set consists of 1595 male blood donors who donated blood multiple times over a period of many years. Each time they visited the donation center, the following information was noted down: Season in which blood was donated (Cold/Hot), Volume of blood donated (ml), age at the time of donation (years), a binary indicator Donate (yes/no) specifying if the donor was allowed to donate the blood, Haemoglobin (Hb) of the patient at the time of donation, and the date of visit. \citet{nasserinejad_predicting_2013,nasserinejad_prevalence_2015,nasserinejad_prediction_2016} have analyzed this data set extensively using transition models, mixed models, growth mixture models and latent class mixed effects transition models to predict the Hb level of donors so that they are invited for donation at an optimal time, i.e. when their Hb levels are not too low because of the previous donations and other factors.\\

 For the purpose of this thesis we did not use the entire data set of 1595 patients as Bayesian computation for heterogenity model using the entire data set required a fairly large amount of computational power. Instead we used a simple random sample of the data set to obtain data of 250 subjects.

\section{Motivation for analysis with Bayesian heterogeneity model}
 In the analysis using Growth mixture models they found 4 different underlying subpopulations in the data set. Firstly, those who have a relatily stable Hb level over donations. Secondly, those who although have a higher Hb level than those in category 1, but show a relatively slow decline of Hb level over donations. Thirdly, those show a moderately sharp decline in Hb level and lastly those who show a steep decline in Hb levels despite beginning at high initial Hb levels. Because of presence of the different subpopulations, a single methodology to decide the time of next blood donation for subjects from all subpopulations may not be effective. The aim of applying the Bayesian heterogeneity model to this dataset is to provide an alternative modeling framework for such data sets.

\section{Frequentist analysis}
\label{sec : frequentist_blood_donor}
We began with a frequentist analysis of the blood donor data set to select the right mean structure for our models ahead. This was crucial as the random effects structure depends on the mean structure. We considered a model with both random intercept and random slope. For the choice of random slope we selected the number of donations in last 2 years as that was deemed as a suitable variable for random slope by \citet{nasserinejad_prevalence_2015}. We found the 