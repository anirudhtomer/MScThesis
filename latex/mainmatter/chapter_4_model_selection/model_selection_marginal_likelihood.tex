% !TEX root =  ../../thesis.tex

\section{Marginal Likelihood}
\label{sec : marginal_likelihood}

The marginal likelihood of data represents the probablilty of data given the model. This can be calculated by margilizing over the model parameters $\boldsymbol{\theta}$. 

\begin{equation}
\label{eq : marginal_likelihood}
p(\boldsymbol{y}|M) = \int p(\boldsymbol{y}|\boldsymbol{\theta}, M) p(\boldsymbol{\theta}|M) \diff \boldsymbol{\theta}
\end{equation}

Given two proposed models for the data, $M_1$ and $M_2$, one can further use the quantity in equation \ref{eq : marginal_likelihood} to calculate model evidence. The idea is to calculate the odds of model $M_1$ against the model $M_2$ given the data. i.e. Posterior odds. This ofcourse means that it is a comparative measure as $\sim M_1 = M_2$. One can write the posterior odds as

$$\frac {p(M_1|\boldsymbol{y})}{p(M_2|\boldsymbol{y})} = \frac {p(\boldsymbol{y}|M_1) p(M_1)} {p(\boldsymbol{y}|M_2) p(M_2)}$$

where $\frac {p(M_1)}{p(M_2)}$ is called prior odds, and $\frac {p(\boldsymbol{y}|M_1)} {p(\boldsymbol{y}|M_2)}$ is called the Bayes Factor.\\

Since we have the same belief in each of these models, prior odds is equal to 1 apriori. To calculate the Bayes Factor we will use the method proposed by \citet{chib_marginal_1995}. Chib's idea is that one can rewrite the Bayes rule in equation \ref{eq : bayes_rule} to get the marginal likelihood formula as

\begin{equation}
\label{eq : chib_trick}
m(\boldsymbol{y}) = p(\boldsymbol{y}|M) = \dfrac {L(\boldsymbol{\theta}|\boldsymbol{y}, M) p(\boldsymbol{\theta}|M)} {p(\boldsymbol{\theta}|\boldsymbol{y}, M)}
\end{equation}

Equation \ref{eq : chib_trick} is valid for all $\boldsymbol{\theta}$, though Chib recommends using posterior mode $\argmax_\theta p(\boldsymbol{\theta}|\boldsymbol{y})$ of parameters or the maximum likelihood estimate $\argmax_\theta L(\boldsymbol{\theta}|\boldsymbol{y})$. We decided to choose the latter of the two. Further, in context of the Bayesian heterogeneity model, we will denote the selected parameter values as ${\boldsymbol{\beta}}^*$, ${\sigma^2}^*$ and $\boldsymbol{\nu}^*$. Thus Chib's approximation for marginal likelihood on log scale is given by,

\begin{equation}
\label{eq : chib_approx}
\log{\hat{m}(\boldsymbol{y})} = \log{L({\boldsymbol{\beta}}^*, {\sigma^2}^*, \boldsymbol{\nu}^*|\boldsymbol{y})} + \log{p({\boldsymbol{\beta}}^*, {\sigma^2}^*, \boldsymbol{\nu}^*)} - \log{p({\boldsymbol{\beta}}^*, {\sigma^2}^*, \boldsymbol{\nu}^*|\boldsymbol{y})}
\end{equation}

Note that we have dropped the model indicator $M$ from equation \ref{eq : chib_approx} for readability. We will now show calculations for determing the marginal likelihood value using Chib's approxmiation.\\ 

Firstly $\log{L({\boldsymbol{\beta}}^*, {\sigma^2}^*, \boldsymbol{\nu}^*|\boldsymbol{y})} = f(\boldsymbol{y}|{\boldsymbol{\beta}}^*, \sigma^2, {\boldsymbol{\nu}}^*)$ can be easily determined using the formula given in equation \ref{eq : obs_data_likelihood}. As for the calculation of $\log{p({\boldsymbol{\beta}}^*, {\sigma^2}^*, \boldsymbol{\nu}^*)}$, it is also straightforward as we take independent priors for these parameters and they are well known in advance. The details of the priors we chose are given in section \ref{subsec : choice_priors}. Assuming that the parameters of component densities of the mixture distribution of random effects are indepdent, one can use the following to calculate $\log{p({\boldsymbol{\beta}}^*, {\sigma^2}^*, \boldsymbol{\nu}^*|\boldsymbol{y})}$.

\begin{multline}
\log{p({\boldsymbol{\beta}}^*, {\sigma^2}^*, \boldsymbol{\nu}^*|\boldsymbol{y})} = 
\sum_{k=1}^K{\log{p(G_k^*|\boldsymbol{y})}} + 
\sum_{k=1}^K{\log{p({\boldsymbol{b}_k^C}^*|G_k^*, \boldsymbol{y})}} + 
\log{p({\boldsymbol{\beta}}^*|G_k^*, {\boldsymbol{b}_k^C}^*, \boldsymbol{y})}\\
+ \log{p({\sigma^2}^*|G_k^*, {\boldsymbol{b}_k^C}^*, {\boldsymbol{\beta}}^*, \boldsymbol{y})} + 
\log{p({\boldsymbol{\eta}}^*|G_k^*, {\boldsymbol{b}_k^C}^*, {\boldsymbol{\beta}}^*, {\sigma^2}^*, \boldsymbol{y})}
\end{multline}

An interesting problem one faces in such an expansion is that the posteriors may not be available as well known density. For e.g. we began with choosing indepdent gamma priors for precision parameter of random effects and uniform prior for correlation. However the posterior density was not well known. One could try to fit it with a wrapper density however as we will show ahead this is practically improbable. An obvious alternative is to choose conjugate priors in such situation. However as we mentioned in section \ref{subsec : choice_priors} the joint conjugate prior in the case of unknown mean and precision matrix is a Normal-Wishart-Prior and the joint posterior is a Normal-Wishart-Posterior. Although one does not use them in practice while using BUGS family of software, the problem of posterior being from a unknown family remains the same. \citet{chib_marginal_1995} suggest using the Rao-Blackwellization method to solve this problem. For e.g. the Rao-Blackwellized estimate of $p(G_k^*|\boldsymbol{y})$ is given by

\begin{equation}
\label{eq : rao_blackwellization_wishart}
\begin{split}
\prod_{k=1}^{K} p(G_k^*|\boldsymbol{y}) & = \int \prod_{k=1}^{K} p(G_k^*| \boldsymbol{y}, \boldsymbol{b}, \boldsymbol{S}, \boldsymbol{b}_k^C) p(\boldsymbol{b}_1^C, \boldsymbol{b}_2^C,..., \boldsymbol{b}_K^C,\boldsymbol{b}, \boldsymbol{S}|\boldsymbol{y}) 
\diff{\boldsymbol{b}_1^C} \diff{\boldsymbol{b}_2^C} ... \diff{\boldsymbol{b}_K^C} \diff{\boldsymbol{b}} \diff{\boldsymbol{S}}\\
& \approx \frac 1 m \sum_{l=1}^m \prod_{k=1}^{K} p(G_k^*| \boldsymbol{y}, \boldsymbol{b}^{(l)}, \boldsymbol{S}^{(l)}, {\boldsymbol{b}_k^C}^{(l)})
\end{split}
\end{equation}

The approximation is \ref{eq : rao_blackwellization_wishart} is done by approximating the integral with the samples obtained from the MCMC iterations.