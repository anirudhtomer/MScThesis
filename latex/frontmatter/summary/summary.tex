% !TEX root =  ../../thesis.tex

\chapter{Summary}
\label{ch : summary}

In this master thesis we fitted a finite mixture distribution for the random effects in a Bayesian linear mixed model. A mixture distribution for random effects allows to model the heterogeneity introduced by ignoring certain covariates in the mean structure of the model or to take into account the unknown non normal distriution for random effects. We then explored effectiveness of Bayesian model selection criteria (DIC, Bayes Factor, PPC) for choosing the number of component densities in the mixture distribution of random effects. Since mixture models are missing data models, we implemented various definitions of DIC as given by \citet{celeux_deviance_2006} for such models. We found that DIC 4 based on complete data likelihood was a fairly good selection criteria. However as the sample size decreased the discerning power of DIC also decreased. We then implemented Bayes Factor based on the approximation given by \citet{chib_marginal_1995} and found that it was not reliable for deciding on number of components required in the model. On the other hand, Posterior predictive checks were a very strong discerning method if indepdent inverse gamma priors were used for variance components, and uniform distribution for correlation, in the distribution of random effects. In regards to the choice of prior distribution for covariance parameters, we found that a Wishart prior for precision matrix(inverse of covariance matrix) overestimates the precision when within subject variance is greater than between subject variance. Thus, it could be a good idea to decrease scale of the intercept and the covariate corresponding to random slope, so that the corresponding variances increase in magnitude.