% !TEX root =  ../../thesis.tex

\chapter{Summary}
\label{ch : summary}

In this master thesis we fitted a finite mixture distribution for the random effects in a Bayesian linear mixed model. Our goal was to evaluate the efficacy of Bayesian model selection methods, namely Deviance information criteria, marginal likelihood and posterior predictive checks for selection of the model with the right number of components in the aforementioned mixture. For this we generated multiple artificial data sets with different types of Gaussian mixture of random effects and applied the model selection criteria on them. Since mixture models are missing data models, we implemented various definitions of DIC as given by \citet{celeux_deviance_2006}. We found that conditional data DIC's which are usually reported by MCMC simulation softwares, are not reliable for selecting the number of mixture components. $\text{DIC}_4$ (section \ref{eq : DIC4}) which was based on complete data likelihood performed the best among all of the DIC's. We recommend using this DIC along with posterior predictive checks (PPC) which also worked very well for detecting overfitting mixtures. We found that if inverse gamma priors were used for variance components, and uniform prior is used for correlation in the distribution of random effects, then PPC's based on such models gave more extreme results in presence of overfitted mixtures of random effects. We also calculated marginal likelihood for the various models using the approximation given by \citet{chib_marginal_1995} and found that it was not reliable for deciding the number of components required in the mixture of random effects. Lastly, we also analyzed the blood donor data set \citep{nasserinejad_prevalence_2015} using Bayesian heterogeneity model. Using DIC and PPC we found that the random effect distribution in this model was a mixture of 2 components.\\