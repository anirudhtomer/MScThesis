% !TEX root =  ../../thesis.tex

\chapter{Preface}
\label{ch : preface}

The following thesis work was conducted as part of the programme completion requirements of MSc. Statistics programme at KU Leuven. When I began working on this project, I had little idea that I would be able to go as far as I have been now. There were many significant obstacles on the way, such as analytical calculations of the various Deviance information criteria definitions, marginal likelihood, choice of posteerior predictive checks, implementating them in a software, label switching and lack of computational power required for this work. However looking back I think it was one of the most interesting project I have done in recent times. The entire work for this thesis has been done using R and JAGS (Just Another Gibbs Sampler). The source code, results of simulations and an electronic draft of this thesis can be found at\\
\url{https://github.com/anirudhtomer/MScThesis}\\

In chapter 1 we present an introduction to mixture distribution and their central role in the formulation of the problem statement for this thesis. In Chapter 2 we present an introduction to the Bayesian paradigm as for the work of this thesis we use Bayesian methods. Futher in Chapter 3 we present the definition of a Bayesian heterogeneity model, and the issues with estimation of parameters in it. In Chapter 4 we present the formulae for various classes of Deviance information crieteria, marginal likelihood and posterior predictive checks that we used for model selection. Chapter 5 includes the results of the simulation study that was performed to check the efficacy of the aforemented Bayesian model selection methods. In chapter 6 we model the Blood donor data set \citep{nasserinejad_prevalence_2015} using a Bayesian heterogeneity model and use results from the simulation study to apply the right model selection criteria.\\

I am grateful to my supervisor Professor Dr. Emmanuel Lesaffre for keeping faith in my capabilities and for guiding me in the right direction. I enjoyed the fact that he never spoonfed me, yet was always available to discuss the difficult parts of the work at hand. He set very clear goals at the beginning of the year and continually monitored my progress thereafter. My interest in Bayesian statistics has grown by magnitudes under his supervision and I am looking forward to contribute more in this area. I would also like to extend my gratitude to Professor Geert Molenberghs and Professor Geert Verbeke for the captivating lectures on longitudinal data analysis. They introduced me to mixed models and empowered me with the tools of trade required to do the frequentist analysis of blood donor data set in this report. I am thankful to Kazem Nasserinejad from ErasmusMC for resolving many of my queries regarding the blood donor data set, and to Igor Milhoranca for providing the much needed inputs at crucial times. Lastly, I am grateful to my parents for the innumerable sacrifices they made to make sure I had as less obstacles as possible during my studies and I dedicate this work to them.\\

Anirudh Tomer\\
Leuven, Belgium