% !TEX root =  ../../thesis.tex
\chapter{Preface}
\label{ch : preface}

The following thesis work was conducted as part of the credits completion requirements of the MSc. Statistics programme at KU Leuven. The problem statement of the thesis is to find out the efficacy of Bayesian model selection criteria, for choosing the right number of components in a mixture distribution of random effects, in a longitudinal model. When I began working on this project, I had little idea that I would be able to go as far as I have been now. There were many significant obstacles on the way, such as derivations of the various definitions of Deviance information criteria and marginal likelihood, and choice of posterior predictive checks. While implementing them I realized that the simulation study for this thesis required much more computational power than I had. Although the pace of execution was slow, yet every time I had a result I was excited to see it. Looking backwards, I think it was perhaps the most interesting project I did in last 3 years.Through and through, I enjoyed every bit of this project. The entire work for this thesis has been done using R and JAGS (Just Another Gibbs Sampler). The source code, results of simulations and an electronic draft of this thesis can be downloaded from:\\
\url{https://github.com/anirudhtomer/MScThesis}\\

In chapter 1 an introduction to mixture distributions and their central role in the formulation of the problem statement for this thesis is presented. Since the project was done using Bayesian methods it became essential to give an introduction to the Bayesian paradigm in Chapter 2. Further in Chapter 3 the definition of a Bayesian heterogeneity model and issues related to the estimation of parameters involved are presented. Chapter 4 constitutes the analytical derivations I did for various classes of Deviance information criteria, marginal likelihood and posterior predictive checks, used in model selection. Chapter 5 includes the results of the simulation study that was performed to check the efficacy of the aforementioned Bayesian model selection methods. In Chapter 6 the results of modeling the Blood donor data set \citep{nasserinejad_prevalence_2015} using a Bayesian heterogeneity model are presented. The results from the simulation study are used to apply the right model selection criteria on the models formulated for the Blood donor data set.\\

I am grateful to my supervisor Professor Dr. Emmanuel Lesaffre for keeping faith in my capabilities and for guiding me in the right direction. I enjoyed the fact that he never spoonfed me, yet was always approachable to discuss the statistical problems. He set very clear goals at the beginning of the year and continually monitored my progress thereafter. My interest in Bayesian statistics has grown by magnitudes under his supervision and I am looking forward to contribute more in this area. I would also like to extend my gratitude to Professor Geert Molenberghs and Professor Geert Verbeke for the captivating lectures on longitudinal data analysis, which empowered me with the tools required for performing the frequentist analysis of blood donor data set. I am thankful to Kazem Nasserinejad from ErasmusMC for resolving many of my queries regarding the blood donor data set, and to Igor Milhorança for providing the much needed inputs at crucial times. Lastly, I am grateful to my parents for the innumerable sacrifices they made to make sure I had as less obstacles as possible during my studies and I dedicate this work to them.\\

Anirudh Tomer